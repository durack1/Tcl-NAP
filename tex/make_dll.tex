%  $Id: make_dll.tex,v 1.7 2006/06/02 07:13:30 dav480 Exp $ 
    % Nap Library: make\_dll.tcl

\section{Interfacing Nap to a DLL based on C or Fortran Code}
    \label{make-dll}

\subsection{Introduction}
    \label{make-dll-Introduction}

The file 
  \texttt{make\_dll.tcl} defines procedures for automatically
  producing an interface from Nap to a 
  \emph{DLL} (\emph{dynamic link library} or 
  \emph{shared library}) based on C or Fortran Code. This process
  defines a new tcl command which can be used either directly or via
  another interface (written in Tcl) defining a Nap function.

\subsection{\texttt{make\_dll}}
\label{make-dll-make-dll}

The standard command used to create a DLL is:
\\
\texttt{make\_dll}
$\mathit{options}$
$\mathit{newCommand}$
$\mathit{argDec}$
$\mathit{argDec}$
$\mathit{argDec}$
$\ldots$
\\
$\mathit{newCommand}$ is name of new command.
\\
Each argument-declaration 
$\mathit{argDec}$ is a list with the form
\\
$\mathit{name}$
$\mathit{dataType}$
$\mathit{intent}$
\\
 where
  \begin{bullets}
    \item 
    $\mathit{name}$ is any string (used only in error messages)
    \item 
    $\mathit{dataType}$ can be: 
    \texttt{c8 i8 u8 i16 u16 i32 u32 f32 f64 void}
    \item 
    $\mathit{intent}$ can be: 
    \texttt{in} (default) or 
    \texttt{inout}. Actual 
    \texttt{in} arguments can be expressions of any type (including
    \texttt{ragged}) and will be converted to the specified type
    (unless this is 
    \texttt{void}).
  \end{bullets}
$\mathit{options}$ are:
\begin{simpleitems}
\item \texttt{-quiet}: Do not echo commands.
\item \texttt{-compile} $\mathit{command}$: C compile-command with options
\item \texttt{-dll} $\mathit{fileName}$: output filename for DLL
	(default: $\mathit{newCommand}$.dll for windows, $\mathit{newCommand}$.so for unix)
\item \texttt{-entry} $\mathit{string}$: User-routine entry-point (default: 
  $\mathit{newCommand}$). Note that fortran entry points often include suffix '\_'.
\item \texttt{-header} $\mathit{fileName}$: header (\texttt{*.h}) filename (default: none)
\item \texttt{-libs} $\mathit{fileNames}$: filenames of extra binary libraries (default: none)
\item \texttt{-link} $\mathit{command}$: Link-command with options
\item \texttt{-object} $\mathit{fileName}$: User-routine object-file
	(default: $\mathit{newCommand}$\texttt{.obj} for windows,
	$\mathit{newCommand}$\texttt{.o} for unix)
\item \texttt{-source} $\mathit{fileName}$: Output file containing C source code of interface
	(default: $\mathit{newCommand}$\texttt{\_i.c})
\item \texttt{-version} $\mathit{n.m}$: Version number (default: \texttt{1.0})
\end{simpleitems}

\subsubsection{C Example}
    \label{make-dll-C-example}

The following example (under SunOS 5.8) defines a new NAP
  function 
  \texttt{partialProd} which calculates partial-products. This is
  analogous to the standard Nap function 
  \texttt{psum} which calculates partial sums. The new function is
  based on the following C file 
  \texttt{pprod.c}:
  \begin{verbatim}
void pprod(int *n, float *x, float *result) {
   int         i;
   float       prod = 1;

   for (i = 0; i < *n; i++) {
       result[i] = prod = prod * x[i];
   }
}
\end{verbatim}

  \begin{verbatim}
% exec cc -c -o pprod.o pprod.c
% make_dll pprod {n i32 in} {x f32} {y f32 inout}
cc -I/sol/home/dav480/tcl/include -c pprod_i.c
ld -G -o libpprod.so pprod_i.o pprod.o  
% load ./libpprod.so                             
% proc partialProd x {
    nap "result = reshape(f32(_), shape(x))"
    pprod  "nels(x)" x result
    nap "result"
}
% [nap "partialProd({2 1.5 3 0.5})"]
2 3 9 4.5
\end{verbatim}

\subsubsection{Fortran 90 Example}
    \label{make-dll-f90-example}

The following fortran 90 example does the same thing as the
  above C example. The f90 source code in the file 
  \texttt{pprod.f90} is:
  \begin{verbatim}
subroutine pprod(n, x, result)
   integer, intent(in) :: n
   real, intent(in) :: x(n)
   real, intent(out) :: result(n)
   integer :: i
   real :: prod
   prod = 1.0
   do i = 1, n
       prod = prod * x(i)
       result(i) = prod
   end do
end subroutine pprod
\end{verbatim}

The following log was produced using the SunOS 5.8 f95 compiler.
  (Note that the entry point is `\texttt{pprod\_}'.)
  \begin{verbatim}
% exec f95 -c pprod.f90
% make_dll -entry pprod_ pprod {n i32 in} {x f32} {y f32 inout}
cc -I/sol/home/dav480/tcl/include -c pprod_i.c
ld -G -o libpprod.so pprod_i.o pprod.o  
% load ./libpprod.so
% proc partialProd x {
    nap "result = reshape(f32(_), shape(x))"
    pprod  "nels(x)" x result
    nap "result"
}
% [nap "partialProd({2 1.5 3 0.5})"]
2 3 9 4.5
\end{verbatim}

\subsection{\texttt{make\_dll\_i}}
    \label{make-dll-make-dll-i}

This procedure is normally used via 
\texttt{make\_dll}, but may be called directly if you prefer to do your own compiling and linking.
It makes the Nap C interface to the user's C function or Fortran subroutine.
It is called as follows:
\\
\texttt{make\_dll\_i}
$\mathit{options}$
$\mathit{newCommand}$
$\mathit{argDec}$
$\mathit{argDec}$
$\mathit{argDec}$
$\ldots$
\\
This command's result is the C code.

The arguments are similar to 
  \texttt{make\_dll}, except that the only options are:
\begin{simpleitems}
  \item \texttt{-entry} $\mathit{string}$: User-routine entry-point (default: 
      $\mathit{newCommand}$). Note that fortran entry points often include suffix '\_'.
  \item \texttt{-header} $\mathit{fileName}$: header (\texttt{*.h}) filename (default: none)
  \item \texttt{-version} $\mathit{n.m}$: Version number (default: \texttt{1.0})
\end{simpleitems}
