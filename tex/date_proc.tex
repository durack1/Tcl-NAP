%  $Id: date_proc.tex,v 1.4 2006/06/07 00:21:24 dav480 Exp $ 
    % Nap Library: date.tcl procedures

\section{Procedures for Formatting Dates and Times}
    \label{date-proc}

\subsection{Introduction}
    \label{date-proc-Introduction}

Date/time concepts are introduced in 
section \ref{date-function-Introduction} (Functions for Formatting Dates and Times).
The file 
  \texttt{date.tcl} defines the two formatting procedures 
  \texttt{format\_jdn} and 
  \texttt{format\_mjd}, as well as these functions.

These two formatting procedures use the standard Tcl 
  \texttt{clock} command. The range of dates which 
  \texttt{clock} can handle depends on the platform. The current
  (8.4.7) Windows version uses a 32-bit signed integer offset (seconds)
  from 1970-01-01, allowing dates from 1901-12-13 to 2038-01-19.

The default formats for both procedures are based on the ISO 8601 standard for dates and times.
This standard is discussed in many places, including
\href{http://en.wikipedia.org/wiki/ISO_8601}{Wikipedia}.

\subsection{Formatting Julian Day Numbers (JDNs)}
    \label{date-proc-format-jdn}

The command
  \\
  \texttt{format\_jdn} $\mathit{jdn}$ [$\mathit{format}$]
  \\formats the Julian Day Number 
  $\mathit{jdn}$, one value per line.

  $\mathit{jdn}$ can be any Nap expression.

  $\mathit{format}$ is that required for the 
  \texttt{clock} command. The default format is 
  \texttt{"\%Y-\%m-\%d"}, which produces a standard ISO 8601
  date in the form `\texttt{$\mathit{yyyy}$-$\mathit{mm}$-$\mathit{dd}$}'.

\subsubsection{Examples}

  \begin{verbatim}
% format_jdn 2453305
2004-10-26
% format_jdn 2453305 "%A %B %d, %Y"
Tuesday October 26, 2004
% format_jdn "{0 7} + date2jdn{2004 10 22}"
2004-10-22
2004-10-29
% format_jdn "{0 7} + date2jdn{2004 10 22}" "%y %m %d"
04 10 22
04 10 29
\end{verbatim}

\subsection{Formatting Modified Julian Dates (MJDs)}
    \label{date-proc-format-mjd}

The command
  \\
  \texttt{format\_mjd} $\mathit{mjd}$ [$\mathit{format}$]
  \\formats the Modified Julian Date 
  $\mathit{mjd}$, one value per line.
  


  $\mathit{mjd}$ can be any Nap expression. This is rounded to the
  nearest second.
  


  $\mathit{format}$ is that required for the 
  \texttt{clock} command. The default format is 
  \texttt{"\%Y-\%m-\%dT\%H:\%M:\%S"}, which produces a standard
  ISO 8601 date/time in the form
`\texttt{$\mathit{yyyy}$-$\mathit{mm}$-$\mathit{dd}$T$\mathit{HH}$:$\mathit{MM}$:$\mathit{SS}$}'.

\subsubsection{Examples}

  \begin{verbatim}
% format_mjd 49797.75
1995-03-21T18:00:00
% format_mjd 49797.75 "%I %p on %A %d %B, %Y."
06 PM on Tuesday 21 March, 1995.
% format_mjd "{0 0.5} + dateTime2mjd{2004 10 22 16 30 59}"
2004-10-22T16:30:59
2004-10-23T04:30:59
% format_mjd "{0 0.5} + dateTime2mjd{2004 10 22 16 30 59}" "%y%m%d %H%M"
041022 1630
041023 0430
\end{verbatim}

