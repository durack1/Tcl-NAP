%  $Id: sample.tex,v 1.9 2006/09/20 08:48:49 dav480 Exp $ 
    % Nap Sample Session

\section{Sample Session}

The following sample session illustrates some basic features of
    Nap. The input command lines begin with the standard Tcl prompt
    `\texttt{\%}'.
    \begin{verbatim}
% nap "x = {2 2.5 5}"
::NAP::13-13
% nap "y = x * x"
::NAP::14-14
% $y
4 6.25 25
\end{verbatim}

The first command assigns to 
    \texttt{x} a vector containing the three elements 2, 2.5 and 5.
    The second command assigns to 
    \texttt{y} a vector containing the three elements which are the
    squares of the corresponding elements of 
    \texttt{x}. The command `\texttt{\$y}' displays the value of 
    \texttt{y}.

Nap stores each variable in memory using a data-structure called
    an 
    \emph{n-dimensional array object} (\emph{NAO}).
    Each NAO has an associated Tcl command called its 
    \emph{object-oriented command} (\emph{OOC}) which is used to
\begin{bullets}
      \item obtain data and other information from the NAO
      \item write data from the NAO to files
      \item modify the NAO.
\end{bullets}
An \emph{OOC-name} (command-name of an OOC) is used
\begin{bullets}
      \item to execute the OOC (like any other Tcl command)
      \item as a unique identifier for the NAO associated with the OOC.
\end{bullets}
An OOC-name has the form 
      `\texttt{::NAP::}\emph{seq}\texttt{-}\emph{slot}', where
\begin{simpleitems}
      \item 
      \texttt{::NAP::} is the Tcl namespace used by NAP
      \item 
      \emph{seq} is the \emph{sequence number} assigned in order of creation
      \item 
      \emph{slot} is the index of an internal table used to provide fast access.
\end{simpleitems}
In the above example, both OOC-names (\texttt{::NAP::13-13} and 
      \texttt{::NAP::14-14}) have slots equal to their sequence
      number, but this is not the case in general since the slots of
      deleted NAOs may be reused.

An assignment (`\texttt{=}') operator has on its left a standard Tcl
    variable name which is assigned the (string) value of the OOC-name.
    Continuing the above example, these string values can be displayed
    using the standard Tcl command 
    \texttt{set}.
\begin{verbatim}
% set x
::NAP::13-13
% set y
::NAP::14-14
\end{verbatim}

    

Thus the command `\texttt{\$y}' is equivalent to the command `\texttt{::NAP::14-14}'. \  Confirming this:
\begin{verbatim}
% ::NAP::14-14
4 6.25 25
\end{verbatim}

If an OOC has no arguments (as above) then it returns the value
    of the NAO (abbreviated if the NAO is large). Arguments can be
    specified as in:
\begin{verbatim}
% $x all
::NAP::13-13  f64  MissingValue: NaN  References: 1  Unit: (NULL)
Dimension 0   Size: 3      Name: (NULL)    Coordinate-variable: (NULL)
Value:
2 2.5 5
\end{verbatim}

This illustrates the `\texttt{all}' 
    \emph{method} (sub-command), which provides a more detailed
    description of the NAO than the default method. The\  following
    example uses the `\texttt{set value}' method to change the value of element
    1 of 
    \texttt{x} from 2.5 to 7. Note that element 1 is the second
    element because the subscript origin is 0 (as in other aspects of
    Tcl) rather than 1 (as in languages such as Fortran).
\begin{verbatim}
% $x set value 7 1
% $x all
::NAP::13-13  f64  MissingValue: NaN  References: 1  Unit: (NULL)
Dimension 0   Size: 3      Name: (NULL)    Coordinate-variable: (NULL)
Value:
2 7 5
\end{verbatim}

    

The similarity between the `\texttt{expr}' and `\texttt{nap}' commands for simple arithmetic 
is shown by:
\begin{verbatim}
% expr "2 * (1 - 0.25)"
1.5
% nap "2 * (1 - 0.25)"
::NAP::25-25
% ::NAP::25-25
1.5
% ::NAP::25-25
invalid command name "::NAP::25-25"
\end{verbatim}

Note that the command `\texttt{::NAP::25-25}' worked the first time but failed
    when it was repeated. The NAOs reference count was zero, as it was
    not referenced by anything (e.g. a Tcl variable). So the NAO and
    its associated OOC were automatically deleted after the first
    execution of the OOC.
    

The need to type the additional command `\texttt{::NAP::25-25}' can be obviated using the Tcl
    bracket (`\texttt{[]}') notation. Tcl executes the bracketed
    command, substitutes its result and then executes the generated
    command. So the above can be replaced by:
    \begin{verbatim}
% [nap "2 * (1 - 0.25)"]
1.5
\end{verbatim}

    

The following example illustrates 
    \emph{array indexing}. The six commands do the following:
    \begin{enumerate}
      \item Assign to the variable 
      \texttt{score} a 32-bit floating-point vector containing the
      five values 56, 75, 47, 99 and 49.
      \item Display 
      \texttt{score}.
      \item Index a vector by a scalar `\texttt{2}' to give a scalar.
      \item Index a vector by a vector `\texttt{\{2 0 4\}}' to give a vector.
      \item Illustrate the operator `\texttt{..}' which defines an 
      \emph{arithmetic progression}.
      \item Use such an arithmetic progression as an index.
    \end{enumerate}
    \begin{verbatim}
% nap "score = f32{56 75 47 99 49}"
::NAP::16-16
% $score all
::NAP::16-16  f32  MissingValue: NaN  References: 1  Unit: (NULL)
Dimension 0   Size: 5      Name: (NULL)    Coordinate-variable: (NULL)
Value:
56 75 47 99 49
% [nap "score(2)"] all
::NAP::20-20  f32  MissingValue: NaN  References: 0  Unit: (NULL)
Value:
47
% [nap "score({2 0 4})"] all
::NAP::25-25  f32  MissingValue: NaN  References: 0  Unit: (NULL)
Dimension 0   Size: 3      Name: (NULL)    Coordinate-variable: (NULL)
Value:
47 56 49
% [nap "0 .. 3"]
0 1 2 3
% [nap "score(0 .. 3)"]
56 75 47 99
\end{verbatim}

The following three commands respectively illustrate:
    \begin{enumerate}
      \item function \texttt{sum}, which has the functionality of mathematical `$\sum$'
      \item function \texttt{count}, which gives the \emph{number of non-missing elements}
      \item the use of these functions to calculate an \emph{arithmetic-mean}
    \end{enumerate}
    \begin{verbatim}
% [nap "sum(score)"]
326
% [nap "count(score)"]
5
% [nap "sum(score) / count(score)"]
65.2
\end{verbatim}

    

The following two commands respectively illustrate:
    \begin{enumerate}
      \item the definition of a tcl procedure to calculate an
      arithmetic-mean using NAP
      \item the calling of this procedure as a Nap function
    \end{enumerate}
    \begin{verbatim}
% proc mean x {nap "sum(x)/count(x)"}
% [nap "mean(score)"]
65.2
\end{verbatim}

    

Procedures defining Nap functions have arguments and results
    which are OOC-names. All the facilities of Tcl and Nap can be used.
    So recursion is allowed, as shown by the following 
    \emph{factorial} example:
    \begin{verbatim}
% proc factorial n {
    if {[[nap "n > 1"]]} {
        nap "n * factorial(n-1)"
    } else {
        nap "1"
    }
}
% [nap "factorial(4)"]
24
\end{verbatim}

    

Note the double brackets (inside braces) in the first line of
    the body of the above procedure. The inner brackets produce an
    OOC-name. The outer brackets execute this OOC to produce the string
`\texttt{0}' (meaning \emph{false}) or `\texttt{1}' (meaning \emph{true}).
