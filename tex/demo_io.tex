%  $Id: demo_io.tex,v 1.4 2006/09/20 08:48:49 dav480 Exp $ 
    % Input/Output

\section{Input/Output}

\subsection{ASCII input/output}

    \begin{verbatim}
% # Create some data to write out
% nap "mat = {
{1.3 -2.9 6.8}
{0.9  5.2 8.1}
}"
::NAP::13-13
% nap "latitude = {-40 -30}"
::NAP::15-15
% nap "longitude = 130 .. 150 ... 10"
::NAP::23-23
% $latitude set unit degrees_north
% $longitude set unit degrees_east
% $mat set unit degC
% $mat set coo latitude longitude
% plot_nao mat -type z
.win1
% 
% # Write to ASCII file 'demo.txt'
% set f [open demo.txt w]
file4
% puts $f [$mat value]
% close $f
% gets_file demo.txt
 1.3 -2.9  6.8
 0.9  5.2  8.1

% # Read matrix from ASCII file
% nap "in = gets_matrix('demo.txt')"
::NAP::4622-4622
% $in a
::NAP::4622-4622  f64  MissingValue: NaN  References: 1  Unit: (NULL)
Dimension 0   Size: 2      Name: (NULL)    Coordinate-variable: (NULL)
Dimension 1   Size: 3      Name: (NULL)    Coordinate-variable: (NULL)
Value:
 1.3 -2.9  6.8
 0.9  5.2  8.1
% 
% # Reading data which does not form matrix
% nap "in = {[gets_file demo.txt]}"
::NAP::4629-4629
% $in a
::NAP::4629-4629  f64  MissingValue: NaN  References: 1  Unit: (NULL)
Dimension 0   Size: 6      Name: (NULL)    Coordinate-variable: (NULL)
Value:
1.3 -2.9 6.8 0.9 5.2 8.1
\end{verbatim}

\subsection{netCDF input/output}

    \begin{verbatim}
% # Create some data to write out (same as in ascii.txt)
% nap "mat = {
{1.3 -2.9 6.8}
{0.9  5.2 8.1}
}"
::NAP::13-13
% nap "latitude = {-40 -30}"
::NAP::15-15
% nap "longitude = 130 .. 150 ... 10"
::NAP::23-23
% $latitude set unit degrees_north
% $longitude set unit degrees_east
% $mat set unit degC
% $mat set coo latitude longitude
% plot_nao mat
.win0
% plot_nao mat -type z
.win1
% 
; # netCDF output to new file written with single command
% $mat netcdf demo.nc temperature
::NAP::4641-4641
;
; # Following will fail if you don't have ncdump installed.
; # You can install it under windows by unzipping file 
; # /sol/home/dav480/tcl_nap/install/windows/netcdf-3.5.0.win32bin.ZIP
; # into say C:\Program Files\netcdf
; # You need to include the bin (e.g. C:\Program Files\netcdf\bin) in 
; # environment variable PATH (which you set using Control Panel: system: advanced).
; 
% ncdump demo.nc
netcdf demo {
dimensions:
        latitude = 2 ;
        longitude = 3 ;
variables:
        int latitude(latitude) ;
                latitude:units = "degrees_north" ;
        int longitude(longitude) ;
                longitude:units = "degrees_east" ;
        double temperature(latitude, longitude) ;
                temperature:_FillValue = nan ;
                temperature:units = "degC" ;
data:

 latitude = -40, -30 ;

 longitude = 130, 140, 150 ;

 temperature =
  1.3, -2.9, 6.8,
  0.9, 5.2, 8.1 ;
}
;
# netCDF input of whole variable
% nap "in = [nap_get netcdf demo.nc temperature]"
::NAP::14406-4855
% $in a
::NAP::14406-4855  f64  MissingValue: NaN  References: 1  Unit: degC
Dimension 0   Size: 2      Name: latitude  Coordinate-variable: ::NAP::14408-5607
Dimension 1   Size: 3      Name: longitude  Coordinate-variable: ::NAP::14409-5114
Value:
 1.3 -2.9  6.8
 0.9  5.2  8.1
% [nap cv(in,0)] a
::NAP::14408-5607  i32  MissingValue: -2147483648  References: 1  Unit: degrees_north
Dimension 0   Size: 2      Name: (NULL)    Coordinate-variable: (NULL)
Value:
-40 -30
% [nap cv(in,1)] a
::NAP::14409-5114  i32  MissingValue: -2147483648  References: 1  Unit: degrees_east
Dimension 0   Size: 3      Name: (NULL)    Coordinate-variable: (NULL)
Value:
130 140 150
;
% # Change row 0 in variable 'temperature' of file 'demo.nc' 
% [nap "{-8 0 -1}"] netcdf -index "0, 0 .. 2" demo.nc temperature
% exec ncdump demo.nc
netcdf demo {
dimensions:
        latitude = 2 ;
        longitude = 3 ;
variables:
        int latitude(latitude) ;
                latitude:units = "degrees_north" ;
        int longitude(longitude) ;
                longitude:units = "degrees_east" ;
        double temperature(latitude, longitude) ;
                temperature:_FillValue = nan ;
                temperature:units = "degC" ;
data:

 latitude = -40, -30 ;

 longitude = 130, 140, 150 ;

 temperature =
  -8, 0, -1,
  0.9, 5.2, 8.1 ;
}
% 
% # Read columns 0 and 2 from variable 'temperature' of file 'demo.nc'
% nap "in = [nap_get netcdf demo.nc temperature ",{0 2}"]"
::NAP::14436-4716
% $in a
::NAP::14436-4716  f64  MissingValue: NaN  References: 1  Unit: degC
Dimension 0   Size: 2      Name: latitude  Coordinate-variable: ::NAP::14441-4715
Dimension 1   Size: 2      Name: longitude  Coordinate-variable: ::NAP::14444-6978
Value:
-8.0 -1.0
 0.9  8.1
\end{verbatim}

\subsection{CIF Input/output}

    \begin{verbatim}
% # Create some data to write out (same as in ascii.txt)
% nap "mat = {
{1.3 -2.9 6.8}
{0.9  5.2 8.1}
}"
::NAP::13-13
% nap "latitude = {-40 -30}"
::NAP::15-15
% nap "longitude = 130 .. 150 ... 10"
::NAP::23-23
% $latitude set unit degrees_north
% $longitude set unit degrees_east
% $mat set unit degC
% $mat set coo latitude longitude
% plot_nao mat
.win0
% plot_nao mat -type z
.win1
% 
; # Write to CIF
% 
% $mat a
::NAP::13-13  f64  MissingValue: NaN  References: 1  Unit: degC
Dimension 0   Size: 2      Name: latitude  Coordinate-variable: ::NAP::15-15
Dimension 1   Size: 3      Name: longitude  Coordinate-variable: ::NAP::1925-1925
Value:
 1.3 -2.9  6.8
 0.9  5.2  8.1
% put_cif mat demo.cif; # Write CIF
% # You can use GUI CIF browser to check this new file 'demo.cif'
% 
% # Read from CIF
% nap "in = [get_cif demo.cif]"
::NAP::17738-5075
% $in a
::NAP::17738-5075  f32  MissingValue: NaN  References: 1  Unit: (NULL)
Dimension 0   Size: 2      Name: latitude  Coordinate-variable: ::NAP::17597-9999
Dimension 1   Size: 3      Name: longitude  Coordinate-variable: ::NAP::17669-9608
Value:
 1.3 -2.9  6.8
 0.9  5.2  8.1
\end{verbatim}

