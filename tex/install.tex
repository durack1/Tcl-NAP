%  $Id: install.tex,v 1.9 2006/09/20 08:48:49 dav480 Exp $ 
    % Installing Tcl/Tk and NAP

\section{Installing Tcl/Tk and Nap}

\subsection{Introduction}
    \label{install-Introduction}

  \emph{Tcl/Tk} and 
  \emph{Nap} are available without charge in both binary (compiled)
  and source-code form. The following instructions explain how to
  install Tcl/Tk and then Nap. The Nap installer for Linux on the
  (64-bit) Intel IA64 installs the total system since this platform is
  not supported by ActiveState.

\subsection{Installing Tcl/Tk}
    \label{install-Installing-Tcl-Tk}

  \href{http://aspn.activestate.com/ASPN//}{ActiveState Corporation} provides free binary distributions of 
  \emph{Tcl/Tk} with many extensions. There are versions for Windows,
  Linux (on Intel 386), SunOS (Solaris) and HP-UX. HTML documentation
  is also available. These can all be downloaded from 
  \href{http://aspn.activestate.com/ASPN/Tcl/Downloads/}{ActiveState Tcl Downloads}.
  
 Read the 
  \href{http://aspn.activestate.com/ASPN/docs/ActiveTcl/at.install.html}{ installation instructions}. Note that these explain how to
  uninstall any existing version of Tcl/Tk, which should be done before
  installing the new one. (If this method of uninstalling fails then it
  is usually best to delete the whole 
  \emph{tcl root directory}.)
  
 The 
  \emph{tcl root directory} is the directory in which Tcl is
  installed. Tcl extensions (e.g. Nap) are usually also installed in
  it. Note that it is denoted below by the variable 
  $tcl\_root$. A typical value for Windows is 
  \texttt{C:$\backslash$Program\ Files$\backslash$Tcl}. Typical values for Unix
  (including Linux) are 
  \verb!~/tcl! (personal installation) and 
  \verb!/usr/local/tcl! (system installation).
  
 Unix users should include 
  $tcl\_root$\texttt{/bin} in the list defined by the standard environment
  variable 
  \texttt{PATH}. If there are problems with dynamic loading then
  try including 
  $tcl\_root$\texttt{/lib} in the list defined by the standard environment
  variable 
  \texttt{LD\_LIBRARY\_PATH}.
  
 The following three executables will be installed in 
  $tcl\_root$\texttt{/bin}:
  \begin{description}
    \item[\texttt{tclsh}]
    This provides a bare-bones command-line 
    \emph{Tcl shell} without Tk. (Tk provides GUI facilities.) It is
    useful for non-interactive applications such as 
\begin{bullets}
    \item Unix shell scripts using Tcl
    \item automatically run (e.g.  \texttt{cron}) jobs using Tcl
\end{bullets}
    \item[\texttt{wish}]
    This 
    \emph{window shell} provides a full Tcl shell including Tk.
    \item[\texttt{tkcon}]
    This provides all the facilities of 
    \texttt{wish}, but in a better console. It is recommended as
    the normal way of running Tcl.
  \end{description}
  
 Each executable will execute a startup script file with the name
  listed in the following table, if this file is found in the home
  directory:

\begin{tabular}{|l||l|l|l|}
    \hline 
      \textbf{Operating System} & 
      \textbf{tclsh} & 
      \textbf{wish} & 
      \textbf{tkcon}
    \\
    \hline 
    \hline 
        \textbf{Windows}
       & 
        \texttt{tclshrc.tcl}
       & 
        \texttt{wishrc.tcl}
       & 
        \texttt{tkcon.cfg}
    \\
    \hline 
        \textbf{Unix}
       & 
        \texttt{.tclshrc}
       & 
        \texttt{.wishrc}
       & 
        \texttt{.tkconrc}
    \\
  \hline
\end{tabular}
  
 If you are unsure of which directory is 
  \emph{home}, then enter the following two commands: 
  \texttt{
  \\cd
  \\pwd}

\subsection{Tcl/Tk Documentation}
    \label{install-tcl-doc}

The Caps/Nap Menu
  \emph{help} menu includes the remote web 
  \href{http://aspn.activestate.com/ASPN/Tcl/Reference/}{Tcl Documentation}.

  The above installation process for Windows installs this
  documentation in the form of the compiled HTML help file 
  \texttt{ActiveTclHelp.chm}. Under Windows, the above help menu
  does include this local copy of the documentation.
  
 The standard installation for Unix installs 
  \texttt{man} pages in the directory 
  $tcl\_root$\texttt{/man}. If you want to use these then include this
  directory in 
  \texttt{MANPATH}.
  
 It is also possible to create a local copy of this documentation
  in HTML form.
Under Unix, this allows the Caps/Nap Menu
  (see section \ref{caps-nap-menu})
 to provide a local
  copy of the Tcl Documentation (similar to that for Windows). To do
  this, download the the HTML documentation into the directory 
  $tcl\_root$\texttt{/doc}. Then unpack the tar file and create a symbolic
  link 
  $tcl\_root$\texttt{/doc/html} pointing to the root directory of the unpacked
  files.
  
 For example, if the Tcl root directory is 
  \verb!~/tcl! then download the version 8.4.9 file 
  \\
  \texttt{ActiveTcl8.4.9.0.121397-html.tar.gz} into the directory 
  \verb!~/tcl/doc/!.
  \\
Change to this directory and then unpack the file using a command such as
  \\
  \texttt{gtar zxf ActiveTcl8.4.9.0.121397-html.tar.gz}
  \\or
  \\
  \texttt{gunzip $<$ gtar zxf ActiveTcl8.4.9.0.121397-html.tar.gz | tar xf -}
  \\Then create the symbolic link using
  \\
  \texttt{ln -s ActiveTcl8.4.9.0.121397-html html}

\subsection{Installing Nap}
    \label{install-Installing-nap}

  \emph{Nap installers} are available for:
\begin{bullets}
    \item Linux on Intel 386
    \item Linux on IA64
    \item SunOS
    \item Windows
\end{bullets}
  
 The 
  \emph{Mac OS X} and 
  \emph{Darwin} platforms are supported via the 
  \href{http://fink.sourceforge.net}{Fink} project, which has a 
  \href{http://fink.sourceforge.net/pdb/package.php/nap}{nap}
  page.
  
 It is hoped to support HP-UX in the near future. The SGI Irix
  platform is no longer supported.
  
 Note that the Linux IA64 installer installs a total tcl system as
  well as Nap. This platform does not have an uninstaller, so it is
  suggested you install into a completely new directory rather than
  over an existing installation. 
For example, if your installation directory is 
\verb!~/tcl! you could do the following:
  \begin{verbatim}
mv ~/tcl ~/tcl_old
mkdir ~/tcl
\end{verbatim}
  
 The above installers can be downloaded from the page selected by clicking
 `Files for Downloading'  on
  \href{http://tcl-nap.sourceforge.net/}
  {http://tcl-nap.sourceforge.net/}.
They are in the form of 
  \emph{starpacks}, which are self-contained binary executables which
  contain the files to be installed. Starpacks are themselves based on
  Tcl and are described in 
  \href{http://www.equi4.com/papers/skpaper1.html}{ \emph{Beyond TclKit - Starkits, Starpacks and other *stuff} }.
  
 Then you simply execute the installer. It will first prompt you
  for the directory in which to install Nap, which is usually the 
  \emph{Tcl root directory}. Then it will check for existing Nap
  files and offer to delete them (strongly recommended) if any are
  found.
  
 Finally it will offer to install three 
  \emph{standard Nap startup script files} (with the names in the
  above table) in your home directory. It is recommended that you
  install these unless you have alternative startup files. Such
  alternative files can include the following commands to facilitate
  use of Nap: 
  \texttt{
  \\package require nap
  \\namespace import ::NAP::*}
  \\If there will be other users then copy the three startup files
  to their home directories.
  
 The script files 
  \verb!~/my_tkcon.cfg! and 
  \verb!~/my.tcl! will be sourced if they exist. You can create 
  \verb!~/my_tkcon.cfg! to tailor the tkcon console to your
  personal requirements. You can create 
  \verb!~/my.tcl! to tailor other aspects of the Tcl system to
  your personal requirements. See section
  \ref{scripts-Sample-Startup-Scripts}
for examples.
  
 These scripts and the standard Nap startup scripts are
  interrelated. The script 
  \texttt{tkcon.cfg} (\texttt{.tkconrc}) sources 
  \verb!~/my_tkcon.cfg! and initialises tkcon to source 
  \texttt{wishrc.tcl} (\texttt{.wishrc}). The script 
  \texttt{wishrc.tcl} (\texttt{.wishrc}) in turn sources 
  \texttt{tclshrc.tcl} (\texttt{.tclshrc}), which sources 
  \verb!~/my.tcl!.
  
 On Unix systems it is possible to install Nap in a directory
  (which we will denote by 
  $nap\_root$) other than the 
  \emph{Tcl root directory}. If 
  $nap\_root$\ $\neq$\  
  $tcl\_root$ then 
  $nap\_root$\texttt{/lib} must be included in the lists defined by:
\begin{bullets}
    \item the standard Tcl variable 
    \texttt{auto\_path}
    \item the standard environment variable 
    \texttt{LD\_LIBRARY\_PATH}.
\end{bullets}
For example, assume Nap is installed in 
  \verb!~/alt!. 
The following two commands are included in 
  \texttt{tclshrc.tcl} or 
  \texttt{.tclshrc}:
\begin{verbatim}
lappend auto_path ~/alt/lib
set env(LD_LIBRARY_PATH) ":~/alt/lib:$env(LD_LIBRARY_PATH)"
\end{verbatim}

\subsection{Nap User's Guide}
    \label{install-nap-doc}

The above installation process installs a copy of the Nap User's Guide
as file
\\
$nap\_root$\texttt{/lib/nap?.?/nap\_users\_guide.pdf}
(where \texttt{?.?} is the version number).
The Caps/Nap Menu (see section \ref{caps-nap-menu})
\emph{help} menu includes this local copy as well as the remote web copy at
\href{http://tcl-nap.sourceforge.net/nap\_users\_guide.pdf}
{http://tcl-nap.sourceforge.net/nap\_users\_guide.pdf}.

\subsection{Nap Source Code}
    \label{install-source}

Source code for Nap is maintained using CVS and can be accessed
  by:
\begin{bullets}
    \item 
      \href{http://tcl-nap.cvs.sourceforge.net/tcl-nap/}{Browsing CVS Source Repository}
    \item 
      \href{http://sourceforge.net/cvs/?group_id=55616}{Using
	a CVS client to access CVS Source Repository}
    \item Downloading tar file from 
    \href{http://sourceforge.net/project/showfiles.php?group_id=55616}{Tcl-Nap Files}.
\end{bullets}

\subsection{Related Packages}
    \label{install-Related-Packages}

Nap now provides the functionality previously provided by the
  (now obsolete) separate packages 
  \emph{convert\_date}, 
  \emph{inform} and 
  \emph{land\_flag}. 
However 
\href{http://www.eoc.csiro.au/cats/caps/}{Caps}
is still separate.
  
 The Windows Nap distribution now includes the 
  \href{http://lbayuk.home.mindspring.com/ezprint/}{ \emph{ezprint} }
	utility, which provides access to printers. No such separate
  printing package is needed for Unix. Previous versions of Nap used
  the 
  \emph{printer} package to provide access to printers on all
  platforms.
  
 Previous versions of Nap provided an interface to the vector
  facility in the package 
  \href{http://sourceforge.net/projects/blt/}{BLT}. This allowed
  use of the BLT 
  \emph{graph} command, which was used at that time by the Nap
  procedure 
  \emph{plot\_nao}. Nap no longer provides an interface to BLT or uses
  it in any way.
