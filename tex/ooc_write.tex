%  $Id: ooc_write.tex,v 1.7 2006/10/27 08:05:37 dav480 Exp $ 
    % OOC Methods: Write

\section{OOC Methods which Write to File}

The examples in this section use the NAOs created by:
  \begin{verbatim}
% nap "x = 0 .. 2 ... 0.5"
::NAP::21-21
% nap "y = x ** 2"
::NAP::24-24
% $y set coord x
% $y set unit m^2
% $y set label "areas of squares"
% $y all
::NAP::24-24  f64  MissingValue: NaN  References: 1  Unit: m^2
areas of squares
Dimension 0   Size: 5      Name: x         Coordinate-variable: ::NAP::21-21
Value:
0 0.25 1 2.25 4
\end{verbatim}

\subsection{Method \texttt{binary}}
    \label{ooc-write-binary}

  $ooc\_name$ \texttt{binary} [$tcl\_channel$]

This writes raw (binary) data to the file specified by 
  $tcl\_channel$, which defaults to 
  \texttt{stdout} (standard output). For example:
  \begin{verbatim}
% set file [open y.tmp w]
file4
% $y binary $file
% close $file
% set file [open y.tmp]
file4
% [nap_get binary $file f64] all
::NAP::248-248  f64  MissingValue: NaN  References: 0  Unit: (NULL)
Dimension 0   Size: 5      Name: (NULL)    Coordinate-variable: (NULL)
Value:
0 0.25 1 2.25 4
% close $file
\end{verbatim}

\subsection{Method \texttt{hdf}}
    \label{ooc-write-hdf}

  $ooc\_name$ \texttt{hdf} [$switches$]  $filename$ $name$

This writes data from the NAO to an SDS or attribute within an HDF
  file named 
  $filename$.
The HDF file format is discussed in section \ref{hdf-netcdf}.

  $name$ is the name of an SDS or attribute and has the
  form:
\begin{bullets}
    \item $varname$ for an SDS
    \item $varname$\texttt{:}$attribute$ for an attribute of an SDS
    \item \texttt{:}$attribute$ for a global attribute
\end{bullets}
  
  $switches$ can be:
  \\
  \texttt{-unlimited}: Create SDS with unlimited dimension 0
  \\
  \texttt{-coordinateVariable} 
  $expr$: boxed NAO which specifies coordinate variables.
  \\
  \texttt{-datatype} 
  $type$: HDF datatype: 
  \texttt{c8, i8, i16, i32, u8, u16, u32, f32} or 
  \texttt{f64}
  \\
  \texttt{-range} 
  $expr$: HDF 
  \texttt{valid\_range}
  \\
  \texttt{-scale} 
  $expr$: HDF 
  \texttt{scale\_factor}
  \\
  \texttt{-offset} 
  $expr$: HDF 
  \texttt{add\_offset}
  \\
  \texttt{-index} 
  $expr$: position within SDS where data is to be written.
  

If `\texttt{-coordinateVariable} 
  $expr$' is not specified then the coordinate variables
  of the main NAO are used if these exist.
  

If `\texttt{-index} $expr$' is specified then 
  $expr$ must be an index compatible with the rank $r$ of the SDS. 
If $r = 1$ then $expr$ must be a {\em shape-preserving} index,
as described in section \ref{indexing-Shape-Preserving}.
If $r > 1$ then $expr$ must be a {\em cross-product} index
(as described in section \ref{indexing-Cross-product-index})
containing $r$ scalar and vector elements.
The vector elements
  correspond to dimensions which exist in both the SDS and the NAO. The
  scalar elements correspond to dimensions which exist in the SDS but
  not the NAO.
  

If `\texttt{-index} 
  $expr$' is not specified then the coordinate variables
  of the main NAO are used if these exist, otherwise writing starts at
  the beginning of the SDS.
  
If `\texttt{-scale} 
  $expr$' and/or `\texttt{-offset} 
  $expr$' are specified then these values are:
\begin{bullets}
    \item written as attributes \texttt{scale\_factor} and \texttt{add\_offset}
    \item used to scale the data before writing it.
\end{bullets}

Example:
  \begin{verbatim}
% $y hdf y.hdf y
% exec hdp dumpsds y.hdf
File name: y.hdf 

Variable Name = y
         Index = 0
         Type= 64-bit floating point
         Ref. = 2
         Rank = 1
         Number of attributes = 3
         Dim0: Name=x
                 Size = 5
                 Scale Type = 64-bit floating point
                 Number of attributes = 0
         Attr0: Name = _FillValue
                 Type = 64-bit floating point 
                 Count= 1
                 Value = DoubleInf 
         Attr1: Name = long_name
                 Type = 8-bit signed char 
                 Count= 16
                 Value = areas of squares
         Attr2: Name = units
                 Type = 8-bit signed char 
                 Count= 3
                 Value = m^2
         Data : 
                0.000000 0.250000 1.000000 2.250000 4.000000 


Dimension Variable Name = x
         Index = 1
         Scale Type= 64-bit floating point
         Ref. = 3
         Rank = 1
         Number of attributes = 0
         Dim0: Name=x
                 Size = 5
         Data : 
                0.000000 0.500000 1.000000 1.500000 2.000000 
\end{verbatim}

Procedure \texttt{put16}
(see section  \ref{bin-io-put16})
 can be used to write an automatically-scaled 16-bit SDS to an HDF file.

\subsection{Method \texttt{netcdf}}
    \label{ooc-write-netcdf}

  $ooc\_name$ \texttt{netcdf} [$switches$]  $filename$ $name$

This writes data from the NAO to a variable or attribute within a
  netCDF file named 
  $filename$.
The netCDF file format is discussed in section \ref{hdf-netcdf}.

  $name$ is the name of a netCDF variable or attribute and has
  the form:
\begin{bullets}
    \item $varname$ for a variable
    \item $varname$\texttt{:}$attribute$ for an attribute of a variable
    \item \texttt{:}$attribute$ for a global attribute
\end{bullets}

  $switches$ can be:
  \\
  \texttt{-unlimited}: Create variable with unlimited dimension 0
  \\
  \texttt{-coordinateVariable} 
  $expr$: boxed NAO which specifies coordinate variables.
  \\
  \texttt{-datatype} 
  $type$: netCDF datatype: 
  \texttt{c8, i16, i32, u8, f32} or 
  \texttt{f64}
  \\
  \texttt{-range} 
  $expr$: netCDF 
  \texttt{valid\_range}
  \\
  \texttt{-scale} 
  $expr$: netCDF 
  \texttt{scale\_factor}
  \\
  \texttt{-offset} 
  $expr$: netCDF 
  \texttt{add\_offset}
  \\
  \texttt{-index} 
  $expr$: position within netCDF variable where data is to be
  written.
  

If `\texttt{-coordinateVariable} 
  $expr$' is not specified then the coordinate variables
  of the main NAO are used if these exist.
  

If `\texttt{-index} 
  $expr$' is specified then 
  $expr$ must be an index compatible with the rank 
  $r$ of the netCDF variable.
If $r = 1$ then $expr$ must be a {\em shape-preserving} index,
as described in section \ref{indexing-Shape-Preserving}.
If $r > 1$ then $expr$ must be a {\em cross-product} index
(as described in section \ref{indexing-Cross-product-index})
containing $r$ scalar and vector elements.
The vector elements
  correspond to dimensions which exist in both the netCDF variable and
  the NAO. The scalar elements correspond to dimensions which exist in
  the netCDF variable but not the NAO.
  

If `\texttt{-index} 
  $expr$' is not specified then the coordinate variables
  of the main NAO are used if these exist, otherwise writing starts at
  the beginning of the netCDF variable.
  

If `\texttt{-scale} 
  $expr$' and/or `\texttt{-offset} 
  $expr$' are specified then these values are:
\begin{bullets}
    \item written as attributes 
    \texttt{scale\_factor} and 
    \texttt{add\_offset}
    \item used to scale the data before writing it.
\end{bullets}

Example:
  \begin{verbatim}
% $y netcdf y.nc y -coordinateVariable "x // {3 6 9}"
% exec ncdump y.nc
netcdf y {
dimensions:
        x = 8 ;
variables:
        double x(x) ;
        double y(x) ;
                y:_FillValue = nan ;
                y:long_name = "areas of squares" ;
                y:units = "m^2" ;
data:

 x = 0, 0.5, 1, 1.5, 2, 3, 6, 9 ;

 y = 0, 0.25, 1, 2.25, 4, _, _, _ ;
}
\end{verbatim}

Note that the netCDF variables \texttt{x} and \texttt{y} 
are dimensioned to 8 (the shape of the argument
  specified by the 
  \mbox{
    \texttt{-coordinateVariable}
  } option).
  
Procedure \texttt{put16}
(see section  \ref{bin-io-put16})
can be used to write an automatically-scaled 16-bit variable to
  a netCDF file.

\subsection{Method \texttt{swap}}
    \label{ooc-write-swap}

  $ooc\_name$ \texttt{swap} [$tcl\_channel$]

Method 
  \texttt{swap} is like method 
  \texttt{binary}, except that bytes are swapped. This enables
  writing of data to be read on a machine with opposite byte-order
  within words.
