%  $Id: syntax.tex,v 1.8 2006/09/21 07:40:12 dav480 Exp $ 
    % Syntax of Nap Expressions

\section{Syntax of Nap Expressions}
    \label{syntax}

\subsection{Introduction}
      \label{syntax-Introduction}

The standard Tcl command\  `\texttt{expr}' is based on C conventions for operators and
    functions. Nap expressions use similar conventions and can include
    any of the following tokens separated by white-space
    characters:
    \begin{itemize}
      \item Operands
      \begin{itemize}
        \item OOC-names
        \item names of Tcl variables (may include namespaces)
        \item constants
        \begin{itemize}
          \item numeric scalar
          \item numeric array
          \item string
        \end{itemize}
      \end{itemize}
      \item Operators (including assignment operator `\texttt{=}')
      \item Parentheses (`\texttt{()}')
      \item Function names
      \begin{itemize}
        \item built-in functions
        \item names (may include namespaces) of Tcl procedures defining
        Nap functions
      \end{itemize}
      \item Tcl substitution characters (`\texttt{[]\$}')
    \end{itemize}

\subsection{Substitution}
      \label{syntax-Substitution}

Like `\texttt{expr}', Nap does the Tcl substitution defined by
    any brackets and dollars (`\texttt{[]\$}') remaining after normal command parsing.
    

However, unlike `\texttt{expr}', Nap also substitutes for Tcl variable
    names that are not preceded by a `\texttt{\$}' (except where the name is the left operand of
    the assignment operator `\texttt{=}'). The value of the Tcl name is treated as a
    Nap expression, which is evaluated and the OOC-name of the result
    replaces the name. This substitution is repeated (up to eight
    times) until a single OOC-name is generated. The expressions in the
    following example include:
\begin{bullets}
      \item Tcl variable 
      \texttt{length} containing the string `\texttt{3.5}'
      \item Tcl variable 
      \texttt{breadth} defined by Nap to contain `\texttt{::NAP::13-13}'
      \item Nap constants 
      \texttt{2} and 
      \texttt{10}
      \item Tcl variable 
      \texttt{area} containing the string `\texttt{length * breadth}'
\end{bullets}
\begin{verbatim}
% set length 3.5
3.5
% nap "breadth = 2"
::NAP::13-13
% [nap "2 * (length + breadth)"]
11
% set area "length * breadth"
length * breadth
% [nap "10 * area"]
70
\end{verbatim}

    

Each constant is replaced by the OOC-name of a NAO representing
    its value. After substitution, the expression consists of
    OOC-names, operators, function names and parentheses.
    

Similarly one can use a Tcl variable as a generic function, as
    in:
    \begin{verbatim}
% set f sqrt
sqrt
% [nap "f 9"]
3
\end{verbatim}

\subsection{Nap Names}
      \label{syntax-Names}

Nap names are used to identify variables and functions.
Nap variables are just Tcl variables pointing to NAOs.
Tcl procedures can be defined to be called as Nap functions.
Nap does allow names to be prefixed by namespace pathnames.

Tcl procedures have always had their own namespaces for the names of variables.
However many packages need to use many global names for both variables and procedures.
Namespaces (introduced into Tcl in version 8.0)
provide a systematic heirarchical (tree) structure for such global names,
just as file directories (folders) provide such a structure for files.
The components of a Tcl name are separated by a double colon (`\texttt{::}') separator.

The following examples illustrate the use of namespaces with Nap.
\begin{verbatim}
% namespace eval ::mySpace {}; # create namespace "mySpace"
% nap "::mySpace::x = 8"
::NAP::13-13
% [nap "3 + ::mySpace::x"]
11
% proc ::mySpace::square x {nap "x * x"}; # Define function
% [nap "::mySpace::square {3 5 2}"];      # Call it
9 25 4
\end{verbatim}

Tcl allows any string to be the name of a variable or a procedure.
The rules for Nap names are more restrictive.
The only valid characters are letters, digits, underscore (`\texttt{\_}') and colons.
Colons are only allowed as double colon (`\texttt{::}') namespace separators.
The final component (excluding the namespace pathname) must commence with either a
letter or an underscore.
If it commences with an underscore it must contain at least one other character.
The components of the namespace pathname must either conform to the same rule or
be unsigned integers.

\subsection{Functions}
      \label{syntax-Functions}

Function arguments can be enclosed by parentheses (as required
    by many other languages), but these parentheses are not required by
    the syntax. A name (which cannot be a Tcl variable name or it would
    have been substituted) followed by an operand (now an OOC-name) is
    treated as a function name. Thus the following two commands are
    equivalent:
    \begin{verbatim}
% [nap "sin(3.14)"]
0.00159265
% [nap "sin 3.14"]
0.00159265
\end{verbatim}
This `sin' function (and `hypot' used below) are described in section \ref{function-Elemental}.

Multiple arguments are separated by commas in the usual way. For
    example:
    \begin{verbatim}
% [nap "hypot(3,4)"]
5
\end{verbatim}

    

A comma is treated as a low-precedence operator which produces a
    \emph{boxed} array. The parentheses 
    \emph{are} needed to force the comma to be executed before the
    function. The fact that comma is just another operator is shown
    by:
\begin{verbatim}
% nap "a = 3"
::NAP::13-13
% nap "b = 4"
::NAP::14-14
% nap "both = a , b"
::NAP::15-15
% $both all
::NAP::15-15  boxed  MissingValue: 0  References: 1
Dimension 0   Size: 2      Name: (NULL)    Coordinate-variable: (NULL)
Value:
13 14
% [nap "hypot both"]
5
\end{verbatim}
Note that the values of `\texttt{both}' are the slot numbers of `\texttt{a}' and `\texttt{b}' .

\subsection{Indexing}
      \label{syntax-Indexing}

Tcl array indices are enclosed by parentheses (`\texttt{()}'), while C uses brackets (`\texttt{[]}'). Nap requires neither, since indexing is
    simply implied by adjacent operands (now OOC-names). Thus the
    following two commands (which give elements 1, 0, 2 and 0 of the
    vector 
    \texttt{\{5 7 6\}}) are equivalent:
    \begin{verbatim}
% [nap "{5 7 6}({1 0 2 0})"]
7 5 6 5
% [nap "{5 7 6}{1 0 2 0}"]
7 5 6 5
\end{verbatim}

\subsection{Parenthesising Function Arguments and Indices}
      \label{syntax-Parenthesising}

As explained above, there is no syntactic need for parentheses
    around single function arguments and array indices. However, since
    most other computer languages do require such parentheses, it may
    aid human readability to include them.
    

Multiple function arguments and cross-product-indexing are both
    defined using a 
    \emph{boxed} argument. A boxed argument is normally produced
    using the comma operator. The comma operator has low precedence, so
    parentheses 
    \emph{are} normally needed for multiple function arguments and
    cross-product-indexing. The following example shows how an element
    of a matrix can be extracted using either 
    \emph{full} or 
    \emph{cross-product} indexing:
    \begin{verbatim}
% nap "matrix = {
{2 3 4}
{5 6 7}
}"
::NAP::30-30
% [nap "matrix{0 2}"]; # Full index
4
% [nap "matrix(0,2)"]; # Cross-product index
4
\end{verbatim}

